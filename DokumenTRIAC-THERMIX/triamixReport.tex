\documentclass[a4paper,11pt]{report}
\usepackage[T1]{fontenc}
\usepackage[utf8]{inputenc}
\usepackage{lmodern,url}
\usepackage{graphicx}
\usepackage{hyperref}
\usepackage{pslatex}
\usepackage{listings}
\usepackage{textcomp}
\usepackage{float}
\usepackage[paper=a4paper,headheight=0pt,left=4cm,top=3cm,right=3cm,bottom=3cm]{geometry}
\usepackage{titling}
\usepackage{pdfpages}
\usepackage{booktabs}
\usepackage[version=4]{mhchem}
\usepackage{isotope}
\usepackage{datetime2}
\usepackage{rotating}
\usepackage{pdflscape}
\usepackage{subfigure}
\DTMsetdatestyle{ddmmyyyy}
\DTMsetup{datesep=--}
\newcommand{\subtitle}[1]{%
  \posttitle{%
    \par\end{center}
    \begin{center}\large#1\end{center}
    \vskip0.5em}%
}
\newcommand{\ra}[1]{\renewcommand{\arraystretch}{#1}}
\newcommand{\addChapter}[1]{\phantomsection \addcontentsline{toc}{chapter}{#1}}
% Tambahkan berkas PDF ke dalam laporan dan gunakan style laporan  
% terhadap berkas ini. 
\newcommand{\inpdf}[1]{
	\includepdf[pages=-,pagecommand={\thispagestyle{fancy}}]{#1.pdf}}
% 
% Tambahkan berkas PDF ke dalam laporan. 
\newcommand{\putpdf}[1]{\includepdf[pages=-]{#1.pdf}}
\renewcommand*\descriptionlabel[1]{\hspace\leftmargin$#1$}
% 
\include{hype.indonesia}

\renewcommand{\contentsname}{Daftar Isi}
\renewcommand{\chaptername}{BAB}
\renewcommand{\bibname}{Daftar Referensi}
\renewcommand{\listfigurename}{Daftar Gambar}
\renewcommand\lstlistlistingname{Daftar Program}
\renewcommand{\figurename}{Gambar}
\renewcommand{\tablename}{Tabel}
%\title{Lampiran II}
%\title{Kajian Komputasi Dinamika Fluida berbasis OpenFOAM}
%\author{Arya Adhyaksa Waskita}
%\date{January 31, 2017}
\begin{document}
\begin{titlepage}

\newcommand{\HRule}{\rule{\linewidth}{0.5mm}} % Defines a new command for the horizontal lines, change thickness here

\center % Center everything on the page


%----------------------------------------------------------------------------------------
%	LOGO SECTION
%----------------------------------------------------------------------------------------

\includegraphics[scale=.25]{pics/logo.png}\\[1cm] % Include a department/university logo - this will require the graphicx package

%----------------------------------------------------------------------------------------
%	TITLE SECTION
%----------------------------------------------------------------------------------------

\HRule \\[0.4cm]
{ \huge \bfseries Dokumen Pengembangan TRIAMIX \\ (TRIso \textit{Analysis Code} coupled with THERMIX capabilities)}\\[0.4cm] % Title of your document
\HRule \\[1.5cm]

%----------------------------------------------------------------------------------------
%	HEADING SECTIONS
%----------------------------------------------------------------------------------------
%\textsc{Sub Bidang Termohidrolika}\\[0.25cm] % Minor heading such as course title
\textsc{Laboratorium Komputasi}\\[0.25cm] % Major heading such as course name
\textsc{\Large Pusat Teknologi dan Keselamatan Reaktor Nuklir}\\[1.5cm] % Name of your university/college

 
%----------------------------------------------------------------------------------------
%	AUTHOR SECTION
%----------------------------------------------------------------------------------------

\begin{minipage}{0.4\textwidth}
\begin{flushleft} \large
\emph{Disusun oleh:}\\
Arya Adhyaksa Waskita
\end{flushleft}
\end{minipage}
~
\begin{minipage}{0.4\textwidth}
\begin{flushright} \large
\emph{Supervisor:} \\
Dr. Eng. Topan Setiadipura
\end{flushright}
\end{minipage}\\[4cm]

% If you don't want a supervisor, uncomment the two lines below and remove the section above
%\Large \emph{Author:}\\
%John \textsc{Smith}\\[3cm] % Your name

%----------------------------------------------------------------------------------------
%	DATE SECTION
%----------------------------------------------------------------------------------------

{\large \today}\\[3cm] % Date, change the \today to a set date if you want to be precise
%{\large 31 Juli 2017}\\[3cm] % Date, change the \today to a set date if you want to be precise
 
%----------------------------------------------------------------------------------------

\vfill % Fill the rest of the page with whitespace

\end{titlepage}

%\tableofcontents

\pagenumbering{roman}
%\maketitle
\clearpage
\setcounter{page}{2}
\addChapter{Daftar Gambar}
\tableofcontents
%\clearpage
\listoffigures
\addChapter{Daftar Program}
\lstlistoflistings
%\clearpage
\pagenumbering{arabic}

\chapter{Pendahuluan}
Analisis keselamatan reaktor nuklir melibatkan sejumlah aspek seperti diperlihatkan pada \figurename~\ref{fig:aspek}. Setelah upaya melakukan rekayasa balik terhadap PANAMA \cite{report1,VERFONDERN201484} untuk aspek kinerja bahan bakar \cite{triac1}, dipandang perlu untuk melanjutkan analisis keselamatan di aspek \textit{thermal hydraulics}.

\begin{figure}[h!]
  \begin{center}
    \includegraphics[scale=.5]{pics/tools.png}
    \caption{Aspek keselamatan reaktor nuklir}
    \label{fig:aspek}
  \end{center}
\end{figure}

Kode komputer THERMIX \cite{vsop1,vsop2} sebagai salah satu kode baku dalam analisis keselamatan reaktor di aspek termal yang turut menghantarkan Jerman sebagai \textit{center of excellent} pada penelitian tersebut. Dari THERMIX, sejarah irradiasi dan kecelakaan yang dialamai partikel triso dapat disimulasikan.

Karenanya, perangkat lunak akan dikembangkan berdasarkan data referensi dan dokumentasi  \textbf{\cite{vsop1,vsop2}}. Hasil rekayasa balik akan berupa prototipe kode komputer/perangkat lunak yang terintegrasi dengan modul analisis keselamatan bahan bakar berbasis partikel triso \cite{triac1} dan analisis ketidakpastian \cite{lhs}.

\chapter{Struktur Program}
Tahapan rekayasa balik dimulai dengan membuka struktur program dan melihat keterkaitan antar fungsi yang terdapat di kode komputer THERMIX. Terdapat 4 program, masing-masing \texttt{THERMIX1.FOR} - \texttt{THERMIX4.FOR}. Subrutin dan fungsi pada masing program tersebut disajikan pada \tablename~\ref{tab:the1} - \tablename~\ref{tab:the2}. Deskripsi yang disajikan merupakan translasi bebas dari Bahasa Jerman menggunakan \href{https://translate.google.com/}{google translate}.

\begin{table}[h!]
  \caption{Daftar fungsi dan subrutin dalam program \texttt{THERMIX1.FOR}}
  \label{tab:the1}

  \begin{center}
    \begin{tabular}{p{3cm}|p{10cm}}
    \toprule
       Fungsi / Subrutin & Deskripsi\\ \midrule
       ABEND & Membuat penanganan kesalahan \\
       BILD & Lembar penciptaan buatan dan halaman akhir \\
       BUBIL & Perhitungan sumber panas konvektif saat ini dan kompensasi komposisi ini. Hanya aktif jika sumber panas dibuat dengan $\alpha * f$ dan \texttt{TFLU} \\
       CALT & Hitung suhu pada kondisi tunak \\
       CALT1 & Menghitung suhu suhu padat yang homogen \\
       CALT2 & Menghitung suhu padat heterous (temperatur zona bola) solusi TRISSIAG dari sistem persamaan penghapusan matriks (GAUSS) \\
       CALT2H & Menghitung suhu padat heterous (temperatur zona bola) solusi sistem persamaan TRIDIAG matriks penghapusan (GAUSS) \\
       CALTA & Menghitung temperatur padat heterous (\textit{stationary billing}) solusi sistem persamaan sebagai SR CALT2 (eliminasi matriks) \\
       CALTAH & Menghitung temperatur padat heterous (\textit{stationary billing}) solusi sistem persamaan sebagai SR CALT2 (eliminasi matriks) \\
       EXPLIZ & Perhitungan eksplisit ke fungsi panas \\
       MAITHX & Program utama THERMIX, 50x80 tingkat perubahan \\
       STEUER & Menetapkan suhu tengah, menciptakan plot waktu, temperatur corr. rangkaian dalam arah y \\ 
       WTSTEU & Kendali penghapusan kinerja di pertukaran panas \\
       \bottomrule
    \end{tabular}
  \end{center}
\end{table}

\begin{table}[h!]
  \caption{Daftar fungsi dan subrutin dalam program \texttt{THERMIX2.FOR}}
  \label{tab:the2}

  \begin{center}
    \begin{tabular}{p{3cm}|p{10cm}} \toprule
    Fungsi / Subrutin & Deskripsi\\ \midrule
    CALT3 & Perhitungan suhu pada heterous (temperatur zona bola) solusi sistem persamaan Gauss-Siedel. Hati-hati menggunakan $\rightarrow$ kapasitas panas*WK APH, tidak bekerja untuk \textit{flash ball} \\
    \bottomrule
    \end{tabular}
  \end{center}
\end{table}


% Daftar Pustaka
\bibliographystyle{IEEEtran}
\bibliography{references}

%\begin{appendix}
%	\include{markLampiran}
%	\setcounter{page}{2}
%	\include{lampiran}
%\end{appendix}

\end{document}
